\documentclass[spanish]{article}

\usepackage{sectsty}
\usepackage{babel}
\usepackage{graphicx}
\usepackage{wrapfig}
\usepackage{booktabs}
\usepackage{longtable}

% Margins
\topmargin=-0.45in
\evensidemargin=0in
\oddsidemargin=0in
\textwidth=6.5in
\textheight=9.0in
\headsep=0.25in

\title{ <VAR>data.titulo</VAR> }
\author{ <VAR>data.autor</VAR> }
\date{\today}

\begin{document}
\maketitle	
\pagebreak

% Optional TOC
\tableofcontents
\pagebreak

%--Paper--

<BLOCK> if data.historico | length > 0 </BLOCK>
\section{INCENDIOS HISTÓRICOS}
<BLOCK> for incendio in data.historico </BLOCK>
\input{<VAR> incendio </VAR>}
<BLOCK> endfor </BLOCK>
<BLOCK> endif </BLOCK>

<BLOCK> if data.helis | length > 0 </BLOCK>
\section{HELICÓPTEROS}
<BLOCK> for heli in data.helis </BLOCK>
\input{<VAR> heli </VAR>}
<BLOCK> endfor </BLOCK>
<BLOCK> endif </BLOCK>

<BLOCK> if data.provincias | length > 0 </BLOCK>
\section{INCENDIOS POR PROVINCIAS}
<BLOCK> for provincia in data.provincias </BLOCK>
\subsection{<VAR> provincia </VAR>}
Los 10 incendios más grandes la provincia de <VAR>provincia</VAR> son los siguientes:
<VAR> incendios.sort_values('sup', ascending=False)[incendios.PROVINCE == provincia ][['FIREDATE', 'COMMUNE', 'sup']][:10].to_latex(index=False, longtable=True) </VAR>
<BLOCK> endfor </BLOCK>
<BLOCK> endif </BLOCK>

\end{document}
